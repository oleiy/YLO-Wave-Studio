\chapter{Opis założeń projektu}
\label{cha:opisZałożeńProjektu}

%---------------------------------------------------------------------------

\section{Cel Projektu}
\label{sec:celProjektu}
Głównym celem projektu jest zaprojektowanie i wdrożenie w pełni responsywnej (RWD) strony internetowej, która ma pełnić funkcję nowoczesnej i profesjonalnej wizytówki Studia YLO Wave.

Strona ma za zadanie efektywnie budować wiarygodność marki poprzez szczegółową prezentację zakresu świadczonych usług, portfolio dotychczasowych realizacji oraz unikalnego wizerunku studia. Docelowo, witryna musi przykuwać uwagę klientów z branży muzycznej – zarówno tych poszukujących miejsca do tworzenia i nagrywania utworów, jak i użytkowników potrzebujących profesjonalnych usług w zakresie postprodukcji i realizacji materiałów gotowych do streamingu.
\section{Zakres Projektu}
\label{sec:ZakresProjektu}
Projekt \textbf{YLO Wave - Studio} jest ściśle ograniczony do realizacji warstwy prezentacji (\textbf{Front-End}). Obejmuje on stworzenie statycznej strony internetowej typu portfolio, zorganizowanej w formie pojedynczej strony z wydzielonymi sekcjami (One-Page).
Projekt obejmuje: 
\begin{itemize} 
      \item Opracowanie kompletnej struktury semantycznej w \textbf{HTML5} dla głównego pliku `index.html`. 
      \item Pełną stylizację interfejsu za pomocą \textbf{CSS3}, wraz z implementacją dynamicznych efektów wizualnych i animacji opartych na CSS. 
      \item Wdrożenie pełnej \textbf{Responsywności (RWD)}, zapewniającej optymalne wyświetlanie na urządzeniach desktopowych i mobilnych. 
      \item Integrację zasobów zewnętrznych, takich jak dedykowane czcionki oraz pliki graficzne zawarte w folderze \texttt{images/}. 
      \item Utworzenie front-endowej struktury formularza kontaktowego, gotowej do późniejszej integracji z warstwą back-endową. 
\end{itemize}

\section{Grupa Docelowa}
\label{sec:GrupaDocelowa}
Strona internetowa YLO Wave - Studio skierowana jest do następujących grup docelowych:
\begin{itemize}
      \item \textbf{Artyści i Zespoły Muzyczne:} Klienci poszukujący profesjonalnego studia do nagrywania wokali, instrumentów, miksowania i masteringu.
      \item \textbf{Producenci Muzyczni:} Specjaliści zainteresowani współpracą w zakresie produkcji muzyki oraz postprodukcji.
      \item \textbf{Użytkownicy Indywidualni:} Osoby potrzebujące usług związanych z profesjonalną realizacją utworu i przygotowaniem go do publikacji na platformach streamingowych (np. Spotify, YouTube).
\end{itemize}

\newpage
\section{Wymagania Funkcjonalne}
\label{sec:WymaganiaFunkcjonalne}
Strona internetowa musi spełniać następujące wymagania funkcjonalne:
\begin{itemize} 
      \item \textbf{Globalne Menu Nawigacyjne:} System musi udostępniać stałe i łatwo dostępne menu, umożliwiające przełączanie się między kluczowymi sekcjami strony (np. Strona główna, Oferta, Projekty, Kontakt).
      \item \textbf{Nawigacja Mobilna:} Na urządzeniach mobilnych, nawigacja musi być prezentowana w formie intuicyjnego, wysuwanego menu (np. Hamburger), które po interakcji wyświetla pełną listę linków.
      \item \textbf{Prezentacja Loga:} Logo YLO Wave musi być stale widoczne w nagłówku i działać jako aktywny link kierujący do początku strony (\texttt{index.html}).
      \item \textbf{Prezentacja Oferty:} Sekcja Oferta musi w czytelny sposób przedstawiać wszystkie świadczone usługi (wraz z opcjonalnym cennikiem lub opisem metodologii wyceny) Studia YLO Wave.
      \item \textbf{Wyświetlanie Portfolio:} Strona musi umożliwiać estetyczną prezentację dotychczasowych projektów (realizacji), budując zaufanie klienta co do jakości świadczonych usług.
      \item \textbf{Prezentacja Studia:} Strona internetowa musi ukazywać wizerunek studia, udostępniając kluczowe informacje wizualne.
      \item \textbf{Formularz Kontaktowy:} Sekcja Kontakt musi zawierać formularz, umożliwiający klientowi łatwe zainicjowanie kontaktu z YLO Wave.
      \item \textbf{Dane Kontaktowe:} Sekcja Kontakt musi zawierać również podstawowe dane, takie jak adres e-mail, numer telefonu oraz adres siedziby studia.
\end{itemize}

\section{Wymagania niefunkcjonalne}
\label{sec:WymaganiaNiefunkcjonalne}
Strona internetowa musi spełniać następujące wymagania niefunkcjonalne:

\begin{itemize}
      \item \textbf{Czas ładowania:} Czas ładowania strony głównej i pozostałych sekcji nie może przekraczać 3 sekund przy standardowym połączeniu internetowym.
      \item \textbf{Stabilność:} Strona musi być statyczna i odporna na błędy związane z dynamicznymi skryptami, ponieważ nie posiada warstwy back-endowej.
      \item \textbf{Skalowalność:} Struktura HTML i CSS musi być na tyle modularna, aby umożliwiała łatwe dodawanie nowych sekcji, zmianę treści oraz być gotowa na dodanie back-endu.
      \item \textbf{Responsywność:} Jest to krytyczne wymaganie: strona musi być w pełni responsywna, dynamicznie dostosowując układ i wielkość elementów do urządzeń desktopowych i mobilnych (smartfony, tablety).
      \item \textbf{Kompatybilność:} Strona musi działać poprawnie i wyświetlać się zgodnie z projektem wizualnym w najnowszych wersjach popularnych przeglądarek internetowych.
      \item \textbf{Estetyka UI:} Interfejs musi być nowoczesny, spójny wizualnie i intuicyjny. Estetyka musi wspierać wizerunek studia YLO Wave.
\end{itemize}
