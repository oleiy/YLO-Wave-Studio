\chapter{Harmonogram realizacji projektu}
\label{cha:harmonogram}

\section{Etapy realizacji}

Proces tworzenia strony internetowej \texttt{YLO Wave - Studio} został zaplanowany jako ciąg kolejnych etapów, obejmujących m.in. analizę wymagań, projektowanie szaty graficznej, kodowanie szkieletu HTML5, implementację stylów CSS oraz responsywności, testowanie oraz przygotowanie dokumentacji technicznej.

Realizacja przebiegała zgodnie z poniższym harmonogramem, jednak w trakcie realizacji konieczne było kilkukrotne cofnięcie się do wcześniejszych faz projektu w celu wprowadzenia poprawek lub dostosowania struktury do nowych funkcjonalności.
Warto podkreślić, że wiele zadań prowadzono równolegle, ponieważ dawało to swobodę realizacji nowych pomysłów. 

Główne fazy projektu to:
\begin{itemize} 
    \item \textbf{Analiza i Planowanie:} Analiza wymagań, definicja celów, wymagań funkcjonalnych i niefunkcjonalnych.
    \item \textbf{Projektowanie}: Tworzenie logo, projektowanie układów i ustalenie identyfikacji wizualnej (kolory, typografia).
    \item \textbf{Implementacja Struktury:} Kodowanie szkieletu HTML5 z wykorzystaniem podejśćia Desktop-First.
    \item \textbf{Implementacja styli i RWD:} Dodawania stylów CSS i wdrożenie responsywności.
    \item \textbf{Testowanie:} Walidacja kodu, testy RWD na różnych urządzeniach i funkcjonalności na różnych przeglądarkach.
    \item \textbf{Dokumentacja:} Opracowanie formalnej dokumentacji technicznej i końcowe podsumowanie projektu.
\end{itemize}

\section{System kontroli wersji i repozytorium projektu}

W trakcie realizacji projektu wykorzystano system kontroli wersji \textbf{Git}, który umożliwił systematyczne zapisywanie kodu.  Do przechowywania i synchronizacji kodu źródłowego użyto platformy \textbf{GitHub}.

Pełne repozytorium projektu \texttt{YLO-Wave-Studio} jest dostępne pod adresem:

\begin{center}
\texttt{https://github.com/oleiy/YLO-Wave-Studio.git}
\end{center}

Repozytorium zawiera ostateczną wersję projektu i dokumentację projektową.