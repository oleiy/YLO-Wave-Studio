\chapter{Podsumowanie Projektu}
\label{sec:summary}

Projekt \texttt{Strona internetowa YLO Wave - Studio} został zrealizowany jako nowoczesna i profesjonalna wizytówka/portfolio, zaimplementowana w całości z wykorzystaniem technologii front-endowych: \texttt{HTML5}, \texttt{CSS3} oraz minimalnego wsparcia \texttt{JavaScript}. Głównym celem było zaprojektowanie w pełni responsywnej (RWD) platformy cyfrowej, mającej za zadanie efektywne budowanie wiarygodności marki YLO Wave i szczegółową prezentację usług studyjnych.

\subsection*{Kluczowe Osiągnięcia i Architektura}

\begin{itemize}
    \item \textbf{Architektura Technologiczna:} Zastosowano semantyczny kod \texttt{HTML5} w strukturze \texttt{One-Page} (pojedynczy plik \texttt{index.html} z wydzielonymi sekcjami) oraz modularny \texttt{CSS}. Do tworzenia zaawansowanych układów wykorzystano natywne moduły \texttt{CSS Flexbox} i \texttt{CSS Grid}, rezygnując z zewnętrznych frameworków, co minimalizuje czas ładowania strony.
    \item \textbf{Responsywność (RWD):} Wdrożono strategię \texttt{Desktop-First}, skalując i upraszczając układ za pomocą zapytań medialnych (\texttt{@media queries}) dla ekranów tabletowych (\texttt{max-width: 1024px}) i mobilnych (\texttt{max-width: 768px}). W kluczowych sekcjach (\texttt{Oferta}, \texttt{Projekty}) układy wielokolumnowe przechodzą na wertykalny stos kart.
    \item \textbf{Estetyka i UI:} Interfejs utrzymany jest w estetyce \texttt{Dark Mode} z wysokim kontrastem. Kluczowym elementem identyfikacji wizualnej jest fioletowy kolor akcentujący, który pełni funkcję wyróżniania elementów interaktywnych i nawiązuje do wizerunku studia \texttt{high-tech}[cite: 1428, 1432, 1440].
    \item \textbf{Interakcja:} Wszystkie elementy interaktywne, takie jak linki w banerze nawigacyjnym, przyciski \texttt{CTA} i karty, zostały wzbogacone o subtelne animacje w stanie (\texttt{hover}) oparte na transformacjach (\texttt{translateY}, \texttt{scale}) i cieniowaniu, co zwiększa intuicyjność interfejsu.
\end{itemize}

Ostateczny efekt spełnia założone cele projektowe, dostarczając wizualnie spójną, szybką i w pełni responsywną wizytówkę cyfrową, gotową na ewentualną przyszłą rozbudowę oraz integrację z warstwą back-endową.chapters/chapter006.tex